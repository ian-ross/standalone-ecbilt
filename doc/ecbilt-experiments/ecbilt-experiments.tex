\documentclass[a4paper,11pt,article]{memoir}

\usepackage[utf8]{inputenc}
\usepackage{fourier}
\usepackage{amsmath,amssymb}

\usepackage{color}
\definecolor{orange}{rgb}{0.75,0.5,0}
\definecolor{magenta}{rgb}{1,0,1}
\definecolor{cyan}{rgb}{0,1,1}
\definecolor{grey}{rgb}{0.25,0.25,0.25}
\newcommand{\outline}[1]{{\color{grey}{\scriptsize #1}}}
\newcommand{\revnote}[1]{{\color{red}\textit{\textbf{#1}}}}
\newcommand{\note}[1]{{\color{blue}\textit{\textbf{#1}}}}
\newcommand{\citenote}[1]{{\color{orange}{[\textit{\textbf{#1}}]}}}

\usepackage{xspace}
\newcommand{\zfive}{\ensuremath{Z_{500}}\xspace}

\usepackage{listings}
\usepackage{courier}
\lstset{basicstyle=\tiny\ttfamily,breaklines=true,language=Haskell}

\usepackage{fancyvrb}
\usepackage{url}

\usepackage{tikz}
\usetikzlibrary{arrows,positioning}

\usepackage{natbib}

\DeclareMathOperator{\diag}{diag}
\DeclareMathOperator*{\argmax}{arg \, max}

\usepackage{subfig}
\usepackage{float}
\newfloat{listing}{tbp}{lop}
\floatname{listing}{Listing}

\title{Standalone ECBILT Experiments}
\author{Ian~Ross}
\date{13 January 2015 -- ???}

\begin{document}

\maketitle

%======================================================================
\chapter{Introduction}

\begin{itemize}
  \item{Build standalone ECBILT.}
  \item{Source of code (standalone vs. version from LOVECLIM).}
  \item{Approach: convert boundary conditions from HadAM3 run to form
    suitable for ECBILT, do same sort of atmosphere-only simulation
    and compare climatology (also looking at model performance along
    the way).}
  \item{First attempt: keep all ECBILT input files the same except for
    SST, sea-ice forcing, which are taken from tcszd UM simulation.}
  \item{Possible second attempt: new land/sea mask and other boundary
    conditions based on UM data.}
  \item{Changing GHG forcing for the original standalone ECBILT isn't
    possible because of the way the LW radiation scheme works: it's an
    empirical fit to LW radiation absorption from the GFDL GCM.  I'm
    not yet sure what GHG conditions that fitting was done against,
    but I'm going to press on assuming that it was pre-industrial, as
    for the UM simulation.}
\end{itemize}


%======================================================================
\chapter{Code setup}

\begin{itemize}
  \item{Build and test system setup.}
  \item{Changes needed to make vanilla ECBILT code build and run.}
\end{itemize}


%======================================================================
\chapter{Boundary conditions}

\begin{itemize}
  \item{ECBILT boundary condition requirements.}
  \item{Choice of UM comparison run.}
\end{itemize}

%----------------------------------------------------------------------
\section{SST and sea-ice only}

\begin{itemize}
  \item{Conversion of boundary conditions (1): SST, sea-ice, GHG
    concentrations, orbital parameters.}
  \item{Conversion of SST: take UM monthly SST fields, Poisson fill
    missing values, regrid to ECBILT grid, mask with ECBILT land/sea
    mask; then use periodic interpolation in time to generate daily
    data.}
\end{itemize}

%----------------------------------------------------------------------
\section{All boundary conditions}

\begin{itemize}
  \item{Conversion of boundary conditions (2): same plus land/sea
    mask, lake mask, orography, albedo.}
  \item{Plots of land/sea mask compared between original ECBILT, UM
    and ECBILT-regridded UM mask.}
\end{itemize}


%======================================================================
\chapter{Model performance}

\begin{itemize}
  \item{Timing.}
  \item{Ideas for performance improvements.}
\end{itemize}


%======================================================================
\chapter{Model climatology}

\begin{itemize}
  \item{Seasonal plots of ECBILT side-by-side with UM results.}
  \item{Surface temperature.}
  \item{Surface pressure.}
  \item{Precipitable water.}
  \item{Precipitation, evaporation and P-E.}
  \item{Low-level/high-level winds with omega.}
  \item{Z500.}
\end{itemize}

%======================================================================
\appendix
\chapter{Experiments}

\subsection*{control-expt-1}

\begin{itemize}
  \item{Original standalone ECBILT model.}
  \item{SST and sea-ice forcing converted from UM \texttt{tcszd}
    experiment.}
  \item{Simulation length: 100 years; elapsed time 75
    min. $\Rightarrow$ 45 elapsed seconds/model year.}
\end{itemize}

\end{document}
