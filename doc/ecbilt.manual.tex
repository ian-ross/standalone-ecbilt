\documentstyle[12pt, titlepage]{article}

%\renewcommand{\baselinestretch}{1.5}
\renewcommand{\textwidth}{14cm}
\renewcommand{\textheight}{22cm}
\renewcommand{\topmargin}{0cm}

\begin{document}

\tableofcontents

\title{Description of Parameter Control and Output Control for the Atmosphere\_Ocean\_Sea-ice Model}
\author{X. Wang\\ 
        \\ {\em KNMI, De Bilt, The Netherlands} \\ 
        \\ {\em version 2} \\ }
\maketitle

%\begin{abstract}

%\end{abstract}
\baselineskip=18pt

\section{Introduction}


This guide is intended to give a general instruction for the users of the climate
model which is developed in the  research section Predictability of Weather and Climate
of the Royal Netherlands Meteorological Institute. This model consists of atmosphere/ocean/sea-ice 
and land components. This guide is the beginning of a series which will  
come out as new features of the model are being developed. Here we describe 
features to set tunable parameters and output control.


One of the characteristics of the model is that we make use of the 'namelist' feature of 
Fortran to be able to update all the tunable model parameters outside
the model code. In other words it is possible to change parameters without recompiling 
the source codes. 

This report contains four parts. In the first part you find a list of all the parameters
which you can update in the namelist. The second part gives an example namelist
showing how to change the parameters. The third part is a list of all 
variables that can be requested for output from the model.
The last part concerns post processing.

\baselineskip=14pt

\newpage
\section{Parameter List}


\noindent 2.1   Input Control Parameters Atmosphere\\ \\
---runatctl  (integration control parameters) -------------------------------------\\ \
\\
\begin{tabular}{llll}
Name	&    Type  &	Variable			&		Default    \\
\\ 
\hline
%----    &    ----   & 	-------			&		-------    \\
nyears  &    integer &	total integration time in years.	&	1  \\
irunlabel&   integer &	number of startfiles inat001.dat                   \\
         &           &  and inoc001.dat	                        &	001\\
iatm	&    integer &	number of atmospheric time steps	&	   \\	
        &            &  in one day.			&	6          \\
nwrskip	&    integer &	number of years between writing                   \\ 
        &            &  the model state to disk.            	&	10 \\	
iadyn	&    integer &	with (1) or without (0) atmospheric	&	   \\	
        &            &  dynamics.			&	1          \\
iaphys	&    integer  &	with (1) or without (0) atmospheric  &             \\
	&	&	physics.			&	1          \\  \hline
\end{tabular} \\
\vspace{1.0cm} \\
\\
----dispar  (dissipation parameters)---------------------------    \\
\\
\begin{tabular}{llll}
Name   &     Type   &	Variable			    &	Default    \\
\\
\hline
tdis   &     real   &	Ekman dissipation time scale at     &              \\
       &	    &	lower level.		            &	3.0        \\
addisl &     real   &	parameter for computing dissipation &              \\
       &            &	time scale for land.		    &   0.5	   \\				
addish &     real   &	parameter for computing dissipation &              \\
       &	    &   time scale over mountains.	    &	0.5        \\
tdif   &    real    &   damping time scale of hyperviscosity&              \\
       &	    &   of the smallest waves at all levels.&	1.0        \\
idif   &    integer &	determines scale-selectivity of                    \\
       &	    &   hyperviscosity: power of Laplace                   \\
       &	    &   operator on PV.			    &	4          \\
trel   &    real    &   temperature relaxation time scale.  &	50.0       \\  \hline
\end{tabular}
\\

\baselineskip=18pt
\newpage 
-----dfmpar  (deformation parameters) ------------------------    \\
\\
\begin{tabular}{llll}
Name    &    Type   &	Variable			    &   Default    \\ 
\\
\hline
\\
rrdef1	&    real   &	Rossby radius of deformation layer 1&	           \\
	&	    &   (200 - 500 hPa). 		    &	0.110      \\
rrdef2	&    real   &	Rossby radius of deformation layer 2               \\ 
	&           &   (500 - 800 hPa).		    &	0.070      \\
h0	&    real   &	mountain scale height.		    &   3.0        \\  \hline       
\end{tabular}
%\vspace{1.5cm} \\
\\

-----moipar  (moisture parameters) ---------------------------      \\
\\
\begin{tabular}{llll}
Name  	&    Type   &	Variable			    &   Default    \\ 
\\
\hline
ihavm	&    integer&	with (1) or without (0) horizontal                 \\
	&	     &  divergence of moisture.		    &   1          \\
ivavm	&    integer &  with (1) or without (0) vertical                   \\
	&	     &  divergence of moisture.		    &   1          \\
imsink	&    integer &  with (1) or without (0) the sources                \\
	&	     &  and sinks of moisture.		    &	1          \\
tdifg   &    real    &  diffusion time scale for moisture                  \\
        &            &  in days                             &   5.0        \\
gpm500  &    real     & mean 500 hPa geopotential height                   \\
        &            &  (in meters) used in the calculation of the         \\
        &             & ground pressure and temperature.    &   5500.0     \\
hmoisr  &    real    &  reduction factor of mountain heights               \\
        &            &  in order to tune the amount of water               \\
        &            &  that is allowed to pass a topographic              \\
        &            &  barrier.                             &   1.0        \\
umoisr  &    real    &  reduction factor of 800 hPa winds                  \\
        &            &  used in advecting the moisture.     &   0.8        \\
rainmax &    real    &  maximum rate of dynamic and convective             \\
        &            &  rain in m/s.                        &   0.000001   \\
hmix    &    real    &	depth of mixed layer of the ocean.  &	50.0       \\
bmoism  &    real    &  maximum bottom moisture.	    &	0.15       \\
evfac	&    real    &	maximum evaporation factor over                    \\   
        &	     &  land.                               &   1.0        \\
relmax	&    real    &	maximum relative humidity before                   \\ 
	&	     &	saturation.  			    &	1.0        \\  \hline       
\end{tabular}
\\
\\
\\

-----fluxpar  (flux parameter) ---------------------------      \\
\\
\begin{tabular}{llll}
Name	&    Type    &	Variable			      &  Default   \\
\\
\hline
cdrag  &    real     &	coefficient in sensible and latent                  \\
       &	     &	air-sea heat flux and wind stress.    &   1.4e-03   \\    
uv10rfx&    real     & reduction factor of 800 hPa winds to                 \\
       &             & obtain 10 meter wind used in the                     \\
       &             & definition of the surface heat flux.   & 0.8         \\ 
uv10m  &    real     & minimum value of 10 meter wind used in               \\
       &             & all surface fluxes.                    & 4.0         \\  
uv10rvs&    real     & reduction factor of 800 hPa winds to                 \\
       &             & obtain 10 meter wind used in definition              \\
       &             & of the wind stress to drive the ocean                \\
       &             & currents.                              & 0.8         \\       
dsnm   &     real    & maximum land snow depth.               & 2000.0      \\
ndayws &     real    & averaging period of wind stress in days.& 30.0       \\ \hline
\end{tabular}
\\
\\

-----forpar  (forcing parameter) ---------------------------      \\
\\
\begin{tabular}{llll}
Name	&    Type    &	Variable			      &  Default   \\
\\
\hline
iartif	&    integer &  with (1 and inf=.true.) or without (0)&            \\
	&            &  artificial forcing.		      &  0         \\
ipvf1	&    integer &  with (1) or without (0) diabatic heating.& 1       \\
ipvf2	&    integer &  with (1) or without (0) advection of  f            \\
	&	     &  by divergent wind.			& 1        \\
ipvf3	&    integer &	with (1) or without (0) stretching term. & 1       \\
ipvf4	&    integer &	with (1) or without (0) advection of               \\
	&	     &	zeta by divergent wind.			& 1        \\
ipvf5	&    integer &	with (1) or without (0) vertical                   \\
	&            &	advection of zeta.                       & 1       \\
ipvf6	&    integer &	with (1) or without (0) solenoidal term. & 1       \\
ipvf7	&    integer &	with (1) or without (0) advection of               \\
	&            &	temperature by divergent wind.		& 0        \\ \hline
\end{tabular}
\newpage

-----radpar   (radiation parameter) ---------      \\
\\
\begin{tabular}{llll}
\\
Name	&    Type  &	Variable		&		Default    \\
\\
solarc	&    real  &	solar constant.		&		1363.0     \\  \hline  
\end{tabular}
\\
\\
\\
\\
-----satfor   (radiation parameter) ---------      \\
\\
\begin{tabular}{llll}
Name	&    Type  &	Variable		&		Default    \\
\\
\hline
nafyear  &   integer  & number of years of data to save.  &    0            \\
isatfor	&    integer  &	save (1) or not to save (0)                         \\
        &             & nafyear of atmospheric data                         \\
        &             & to disk to be used to drive                         \\
        &             & the ocean in uncoupled mode.      &    0            \\
nbsatfor &   integer  & first year in the integrations to                   \\
         &            & start saving nafyear  of data.    &    0            \\
irunatm  &   integer  & =1, if the atmospheric model is run.                \\
         &            & =0, if the ocean is forced with                     \\
         &            & atmospheric data reading from the disk. & 1         \\  \hline
\end{tabular}
\\
\\
\\
\\
%\newpage
\noindent 2.2  Input Control Parameters Ocean\\ \\
---runoctl  (integration control parameters) -------\\ \
\\
\begin{tabular}{llll}
Name	&    Type  &	Variable			&		Default    \\
\\ 
\hline
\\
idtbtrop&    integer &	length of barotropic timestep           &          \\
        &            &  of ocean in days.	                &      5   \\	
idtbclin&    integer &	length of baroclinic timestep           &          \\
        &            &  of ocean in days.	                &      5   \\  \hline 	
\end{tabular} \\
\\
\\
\newpage
---couppar  (coupling parameters between atmosphere and ocean ) -------\\ \
\\
\begin{tabular}{llll}
Name	&    Type  &	Variable		&		Default    \\
\\ 
\hline
%----   &    ----    & 	-------			&		-------   \\
icoupleh&    integer &	with (1) or without (0) heat flux       &         \\
        &            &  exchange between atmosphere and ocean.  &     1   \\
icouples&    integer &	with (1) or without (0) salinity        &         \\
        &            &  exchange between atmosphere and ocean.  &     1   \\	
icouplew&    integer &	with (1) or without (0) wind stress     &         \\
       	&	     &  forcing from atmosphere to ocean.       &     1   \\
ltflux  &    logical &  true:  prescribed heat flux       	&         \\
        &            &  false: relaxation               	&     F   \\	
mico    &    logical &  true:  prescribed salt flux      	&         \\
        &            &  false: relaxation        		&     F   \\	
ndreloct&    real    &	relaxation constant for temperature     &         \\
        &            &  in days.                                &    50   \\
ndrelocs&    real    &	relaxation constant for salinity        &         \\
        &            &  in days.                                &    50   \\  \hline
\end{tabular} \\
\\
\\
---icepar  (sea-ice parameters) --------------\\ \
\\
\begin{tabular}{llll}
Name	&    Type  &	Variable			&		Default    \\
\\ 
\hline
rks     &    real    &	thermal conductivity of snow.   	&      0.3097 \\
rki     &    real    &	thermal conductivity of ice.     	&      2.034  \\
abottom &    real    &	thermal exchange coefficient            &             \\
        &            &  between ocean and sea-ice.              &      20.0   \\  \hline	
\end{tabular} \\
\\
\\
---ocpar  (ocean parameters) --------------\\ \
\\
\begin{tabular}{llll}
Name	&    Type  &	Variable			&		Default    \\
\\ 
\hline
rkap    &    real    &	Stommel friction.	                &      8e-6   \\	
rkapv   &    real    &	vertical diffusion coefficient.	        &      3e-5   \\	
rkaph   &    real    &	horizontal diffusion coefficient.	&      1.0e+3 \\     	
accstr  &    real    &	strength of acc current.           	&      100e+6 \\ \hline
\end{tabular} \\
\\
\newpage
---levpar  (depth of ocean level parameters ) --------------\\ \
\\
\begin{tabular}{llll}
Name	&    Type  &	Variable			    &		Default    \\
\\ 
\hline
h(1)    &    real    &	depth of 1st ocean level.   	            &          30.0   \\
h(2)    &    real    &	depth of 2nd ocean level.     	            &          50.0   \\
h(3)    &    real    &	depth of 3rd ocean level.                   &          80.0   \\	
h(4)    &    real    &	depth of 4th ocean level.	            &          140.0  \\	
h(5)    &    real    &	depth of 5th ocean level.	            &          250.0  \\	
h(6)    &    real    &	depth of 6th ocean level.	            &          350.0  \\     	
h(7)    &    real    &	depth of 7th ocean level.                   &          400.0  \\ 
h(8)    &    real    &	depth of 8th ocean level.   	            &          450.0  \\
h(9)    &    real    &	depth of 9th ocean level.     	            &          500.0  \\
h(10)   &    real    &	depth of 10th ocean level.                  &          550.0  \\	
h(11)   &    real    &	depth of 11th ocean level.	            &          600.0  \\	
h(12)   &    real    &	depth of 12th ocean level.	            &          600.0  \\ \hline	
\end{tabular} \\
\\
\\
\noindent 2.3   Output Control Parameters Atmosphere\\ \\
-----outatctl  (output control parameters) -------------------------     \\
\begin{tabular}{llll}
\\
Name	 &   Type&	Variable			&	Default    \\
\\
\hline
ioutdaily&   integer&	output (1) or not (0) of instan-   &               \\
	&           &	taneous fields.			&	0          \\
ixout	&    integer&	output frequency of instantaneous                  \\
	&	    &	fields (in days).	        &	30         \\
meantype &   integer&	control output mean type:                          \\
	 &	    &	= 0 no mean will be computed                       \\
	 &	    &   = 1 monthly mean                                   \\
	 &          &	= 2 seasonal mean.      	&	1          \\
meanyl	 &   integer&	output (1) or not (0) of monthly                   \\
	 &	    &	or seasonal mean fields                            \\
	 &          &	depending on meantype.		&	1          \\	
meantot	 &   integer&	output (1) or not (0) of whole-period              \\
	 &	    &	monthly or seasonal mean fields                    \\
	 &	    &	depending on meantype.	 	&	0          \\
ifrendat &   integer&	frequency for writing restart data                 \\
	 &	    &   in days.			&	30         \\  \hline
\end{tabular}
\\
\newpage
------wratpar  (array for selecting output atmospheric variables)  ------------          \\
\\
\begin{tabular}{llll}
Name	&    Type          &	Variable			        &	Default    \\
\hline
t       &    integer array &    upper level temperature		&	0, 0, 0    \\
u	&    ..	           &    wind in x direction		&	0, 0, 0    \\
v	&    ..            &    wind in y direction		&	0, 0, 0    \\
q	&    ..	           &	specific humidity		&	0, 0, 0    \\
omega	&    ..	           &    wind in z direction (in pressure &                 \\
	&	           &	vertical coordinate)		&	0, 0, 0    \\
ts	&    ..            &	surface temperature		&	0, 0, 0    \\
bm	&    ..            &    bottom moisture          	&	0, 0, 0    \\
sdl	&    ..	           &	snow depth over land           	&	0, 0, 0    \\
lsp	&    ..	           &	large scale precipitation	&	0, 0, 0    \\
cp	&    ..	           &	convective precipitation	&	0, 0, 0    \\
shf	&    ..	           &	surface sensible heat flux      &	0, 0, 0    \\
lhf     &    ..	           &	surface latent heat flux        &	0, 0, 0    \\
psi	&    ..	           &	stream function			&	0, 0, 0    \\
chi	&    ..	           &	velocity potential		&	0, 0, 0    \\
r	&    ..	           &	relative humidity		&	0, 0, 0    \\
runoffl	&    ..	           &	runoff over land		&	0, 0, 0    \\
runoffo	&    ..	           &	runoff over ocean		&	0, 0, 0    \\
uv10	&    ..	           &    10 meter height wind magnitude  &	0, 0, 0    \\
t2m     &    ..            &    temperature at 2m height	&	0, 0, 0    \\
alb	&    ..	           &	surface albedo           	&	0, 0, 0    \\
ssr	&    ..	           &	surface solar radiation		&	0, 0, 0    \\
tsr	&    ..	           &	top solar radiation	        &	0, 0, 0    \\
str	&    ..	           &	surface thermal radiation	&	0, 0, 0    \\
ttr	&    ..	           &	top thermal radiation	        &	0, 0, 0    \\
ustress	&    ..	           &	u wind stress			&	0, 0, 0    \\
vstress	&    ..	           &	v wind stress			&	0, 0, 0    \\
evap	&    ..	           &	evaporation			&	0, 0, 0    \\
omegas	&    ..            &    vertical wind at the surface	&	0, 0, 0    \\
pp	&    ..            &	large scale precipitation +     &                  \\
	&	           &	convective precipitation	&	0, 0, 0    \\
hforc   &    ..   	   &    heating force			&       0, 0, 0    \\
vforc   &    ..            &    potential vorticity forcing	&       0, 0, 0    \\
ageu	&    ..	           &	ageostrophic wind in x direction&	0, 0, 0    \\	    
agev	&    ..	           &	ageostrophic wind in y direction&	0, 0, 0    \\
sp  	&    ..	           &	surface pressure                &	0, 0, 0    \\
ttc  	&    ..	           &	total cloud cover               &	0, 0, 0    \\
palb  	&    ..	           &	planetary albedo                &	0, 0, 0    \\
eminp	&    ..	           &	evaporation - precipitation	&	0, 0, 0    \\  \hline   
\hline
\end{tabular}

\newpage

\noindent 2.4  Output Control Parameters Ocean\\ \\
------outoctl (output control parameter)  ------------          \\
\\
\begin{tabular}{llll}
Name	  &    Type          &	 Variable			  &	Default    \\ 
\\
\hline
isnapshot &    ..            &   output (1) or not (0) instan-    &          \\          
          &                  &   taneous fields.                  &       0  \\ 
ifreq     &    ..            &   frequency of outputting instan-  &          \\ 
          &                  &   taneous fields in days.          &	  30 \\
mtype     &    ..            &   =0, no output of mean fields.                \\ 
          &                  &   =1, output monthly mean fields.             \\
          &                  &   =2, output seasonal mean fields. &	  1  \\  \hline
\end{tabular}
\\
\\
\\
------wrocpar (array for selecting output ocean variables)  ------------          \\
\\
\begin{tabular}{llll}
Name	&    Type          &	Variable			        &	Default    \\
\\
\hline
to	&    integer array  &	ocean temperature.		                &	0, 0 \\
sa      &    ..            &    ocean salinity.		                        &	0, 0  \\
ro      &    ..            &    sea water density.       	                &	0, 0  \\
hadvt   &    ..            &    horizontal (x+y) advection of temperature.	&       0, 0  \\
hadvs   &    ..            &    horizontal (x+y) advection of salinity.    	&       0, 0  \\
vadvt   &    ..            &    vertical advection of temperature.	        &       0, 0  \\
vadvs   &    ..            &    vertical advection of salinity.               	&       0, 0  \\
hdift   &    ..            &    horizontal (x+y) diffusion of temperature.	&       0, 0  \\
hdifs   &    ..            &    horizontal (x+y) diffusion of salinity.  	&       0, 0  \\
vdift   &    ..            &    vertical diffusion of temperature.	        &       0, 0  \\
vdifs   &    ..            &    vertical diffusion of salinity.                 &       0, 0  \\
convt   &    ..            &    convective adjustment of temperature.  	        &       0, 0  \\
convs   &    ..            &    convective adjustment of salinity.	        &       0, 0  \\
relaxt  &    ..            &    heat forcing of the ocean.                      &       0, 0  \\
relaxs  &    ..            &    salinity forcing of the ocean.                  &       0, 0  \\
uo	&    ..	           &    u velocity of ocean circulation.                &	0, 0  \\
vo	&    ..            &    v velocity of ocean circulation.                &	0, 0  \\
wo	&    ..	           &    w velocity of ocean circulation.                &	0, 0  \\
heex    &    ..            &    heat flux from atmosphere to ocean.             &       0, 0  \\
saex    &    ..            &    fresh water flux from atmosphere to ocean.      &       0, 0  \\
brin    &    ..            &    fresh water flux due to ice change.             &	0, 0  \\
heico   &    ..            &    heat flux from ice to ocean.                    &       0, 0  \\
dice    &    ..            &    ice thickness.                                  &       0, 0  \\ 
sds     &    ..            &    snow depth over sea-ice.                        &	0, 0  \\
tice    &    ..            &    temperature of ice.                             &	0, 0  \\ \hline
\end{tabular}
\\
\baselineskip=14pt
	    
	    
\newpage
\section{Example Namelist}
Here you find an example namelist. Such a file is part of the model. When 
the model is run it reads the namelist in. The values of the parameters in the namelist 
overrule the default values set in the model. In this way you may modify parameter values
without compiling the source code again. The order of the parameters in each block of the
namelist is arbitrary.
In this namelist the first 13 blocks are input control parameters (see chapter 2). They
are self explanatory. The last 4 blocks are output control parameters.
After the example namelist a brief explanation about output control parameters is
given.  \\
%%
\\
\\
\\
\begin{tabular}{lll}
begin namelist      \\
                    \\
\$runatctl          \\
 nyears    & = & 2  \\
 irunlabel & = & 15 \\
 iatm      & = & 6  \\
 nwrskip   & = & 5  \\
\$end               \\
                    \\
\$dispar            \\
 tdis   &  = & 3.0  \\
 addisl &  = & 0.5  \\
 addish &  = & 0.5  \\
 trel   &  = & 50.0 \\
 tdif   &  = & 3.0  \\
 idif   &  = & 2    \\
\$end               \\
                    \\
\$dfmpar            \\
 rrdef1 &  = &  0.120 \\
 rrdef2 &  = &  0.070 \\
 h0     &  = &  3.    \\
\$end                 \\
\end{tabular}

\newpage
%%%
\begin{tabular}{lll}
                          \\
\$moipar             \\
 ihavm    & = & 1    \\
 ivavm    & = & 1    \\
 imsink   & = & 1    \\
 tdifq    & = & 2d0  \\
  umoisr   & = & 0.6  \\
 rainmax  & = & 1e-5 \\
\$end                \\
                            \\
\$fluxpar              \\
 cdrag    & = & 1.4e-03\\
 uv10rfx  & = & 0.8    \\
 uv10m    & = & 4.     \\
\$end                  \\
                            \\
\$forpar          \\
 iartif &   = & 0 \\
 ipvf1  &   = & 1 \\
 ipvf2  &   = & 1 \\
 ipvf3  &   = & 1 \\
 ipvf4  &   = & 1 \\
 ipvf5  &   = & 1 \\
 ipvf6  &   = & 1 \\
 ipvf7  &   = & 1 \\
\$end             \\
                           \\
\$radpar          \\
\$end             \\
                           \\
\$satfor          \\
 isatfor &  = & 1 \\
 nbsatfor&  = & 5 \\
 nafyear &  = & 1 \\
 irunatm &  = & 1 \\
\$end             \\
\end{tabular}
%%%%% 
\newpage                 
\begin{tabular}{llll}
                  \\
\$runoctl         \\
 idtbclin &  = 1  \\
 idtbtrop &  = 1  \\
\$end             \\
                                \\
\$couppar         \\
 ndreloct &  = & 20 \\
 ndrelocs &  = & 40 \\
 icoupleh &  = & 1  \\
 icouplew &  = & 1  \\
 icouples &  = & 1  \\
 ltflux   &  = & .false. \\
 mico     &  = & .false. \\
\$end             \\
                                \\
\$icepar           \\
 rks      &  = &  0.3097 \\
 rki      &  = &  2.034  \\
 abottom  &  = &  20d0  \\
\$end             \\
                                \\
\$ocpar           \\
 rkap    &  = &  8d-6  \\
 rkapv   &  = &  3d-5  \\
 rkaph   &  = &  1d3   \\
 accstr  &  = &  100d6 \\
\$end             \\
                                \\
\$levpar          \\
\$end             \\
                                \\
\$outatctl&       \\
  ioutdaily &   = & 1    \\
  ixout     &   = & 5    \\
  meantype  &   = & 1    \\
  meantot   &   = & 1    \\
  meanyl    &   = & 0    \\
\$end                    \\
\end{tabular}
%%%%%
\newpage
\begin{tabular}{llll}
                         \\
\$wratpar                 \\
 ts    &  = & 0,      0,          2 \\
 t     &  = & 0,      1, 	  1 \\
 u     &  = & 0,      2,	  0 \\
 v     &  = & 0,      2,	  0 \\
 omega &  = & 0,      2,	  0 \\
 psi   &  = & 1,      0,	  0 \\
 shf   &  = & 0,      0,	  2 \\
 lhf   &  = & 0,      0,	  2 \\
 lsp   &  = & 0,      0, 	  0 \\
 hforc &  = & 0,      0,          0 \\
 vforc &  = & 0,      0,          0 \\
 cp    &  = & 0,      0, 	  0 \\
 q     &  = & 0,      0,          1 \\
 r     &  = & 0,      0,          1 \\
 ageu  &  = & 0,      0,          1 \\
 agev  &  = & 0,      0,          1 \\
 ssr   &  = & 0,      0,          0 \\
 tsr   &  = & 0,      0,          0 \\
 ttr   &  = & 0,      0,          0 \\
 str   &  = & 0,      0,          0 \\
 bm    &  = & 0,      0, 	  0 \\
 pp    &  = & 0,      0, 	  0 \\
 evap  &  = & 0,      0,	  0 \\
 eminp &  = & 0,      0,	  0 \\
 albs  &  = & 0,      0,	  0 \\
 ustress  &  = & 0,      0,	  0 \\
 vstress  &  = & 0,      0,	  0 \\
 sdl      &  = & 0,      0,	  0 \\
 uv10     &  = & 0,      0,	  0 \\
 runoffo  &  = & 0,      0,	  0 \\
 runoffl  &  = & 0,      0,	  0 \\
 chi      &  = & 0,      0,	  0 \\
 sp       &  = & 0,      0,	  0 \\
\$end                    \\
\end{tabular}
%%%%
\newpage
\begin{tabular}{llll}
                         \\ 
\$outoctl&               \\
  isnapshot &   = & 1    \\
  ifreq     &   = & 90   \\
  mtype     &   = & 1    \\
\$end                    \\
                         \\
\$wrocpar        \\
 to      &  = & 1,  1 \\
 sa      &  = & 1,  1 \\
 uo      &  = & 1,  1 \\
 vo      &  = & 1,  1 \\
 wo      &  = & 1,  0 \\
 relaxt  &  = & 0,  2 \\
 relaxs  &  = & 0,  2 \\
 heex    &  = & 0,  0 \\
 saex    &  = & 0,  0 \\
 brin    &  = & 0,  0 \\
 tice    &  = & 0,  0 \\
 dice    &  = & 0,  0 \\
 heico   &  = & 0,  0 \\
\$end  
\end{tabular}
%%%%%
\\
end namelist
\baselineskip=18pt

The parameters in block \$outatctl are the general control parameters for the atmospheric output. 
The meaning of each parameter should be clear from chapter 2.

The parameters in \$wratpar are used to select each individual atmospheric variable. They are integer
arrays. The first element of each array relates to instantaneous fields. The second 
relates to the whole-period monthly or seasonal mean and standard deviation. The third element 
relates to the year-to-year monthly or seasonal mean and standard deviation.\\
\\
\begin{tabular}{ll}
  A value 0 always means NO output.  \\ 
  A value 1 means output of instantaneous or mean fields. \\  
  A value 2 (only possible for elements 2 and 3) means output of the mean fields \\
together with the standard deviation around it.
\end{tabular}
\\
\\
In this example namelist we see from \$outatctl that:

\newpage
Instantaneous fields are wanted with an output frequency of 5 days. From the first elements of
the arrays in \$wratpar we see that only psi will be outputted.

Meantype is 1 in \$outatctl. So the monthly-mean will be outputted.
Meantot is 1 means that the monthly-mean will be computed over the whole period. The second element of each 
array in the \$wratpar block tells us which variables are requested. We see that
t is selected for the mean only and u,v,omega are selected for the mean and 
the standard deviation and all the other variables are not selected at all.

If meanyl were 1, then monthly-means would also be computed for each year.               
But because here meanyl is set to 0, the selection in \$wratpar as indicated by the third elements
of the arrays (ts, shf and lhf are selected for mean and standard deviation and 
t is selected for mean only) has no effect. 

For ocean output we see from \$outoctl that the instantaneous fields are wanted because isnapshot is set to 
1. The output frequency is set to 90 days. 
The selection of each individual variable is given by the first
elements of the arrays in \$wrocpar. 
We see that the ocean temperature, salinity, u velocity, v velocity and w velocity are selected here.  We see that 
mtype is equal to 1 and therefore monthly mean fields are computed. 
For each individual variable this is decided by the second elements of the arrays in \$wrocpar.
In this example we see that the ocean temperature, salinity, u velocity, v velocity are selected for mean only while 
relaxt and relaxs are selected for mean and standard deviation. All the other variables of the ocean are not selected
at all.
\section{Variables} 
\noindent{4.1   Atmospheric Variables}
\\
The following variables can be requested for output from the climate model (see Table-1).  The code number is according 
to ECMWF MARS convention. Some model-specific variables have self invented code numbers. The variables of the
atmospheric model can be divided into two categories.  One are defined on T-levels, the other on U-levels.
T-level pressure values are 1000 hPa, 650 hPa and 350 hPa. U-level pressure values are 850 hPa, 500 hPa 
and 250 hPa. 
\newpage
The capital T and U in brackets following the numbers of levels for each variable indicate to 
which kind of level it belongs. The necessity of making this distinction will become clear in chapter 5. \\
\\ 
\begin{tabular}{lllll} 
name in & code   & level   &variable		  	&    unit           \\ 
namelist  \\ 
\hline 
 t     &  130   &  3(T)	&temperature.		        &   ($^oC$)         \\ 
 u     &  131   &  3(U)	&wind in x direction.		&    ($m/s$)        \\
 v     &  132   &  3(U)	&wind in y direction.		&    ($m/s$)        \\
 q     &  133   &  1(T)	&specific humidity.		&    ($kg/kg$)      \\
 omega &  135   &  3(T)	&vertical pressure wind.	&    ($Pa/s$)       \\
 ts    &  139   &  1(T)	&surface temperature.		&    ($^oC$)        \\
 bm    &  140   &  1(T)	&bottom moisture.               &    ($m$)          \\
 sdl    & 141   &  1(T)	&snow depth over land.          &    ($m$ )         \\
 lsp  &   142   &  1(T)	&large scale precipitation.     &    ($cm/year$)    \\
 cp   &   143   &  1(T)	&convective precipitation.	&    ($cm/year$)    \\
 shf  &   146   &  1(T)	&surface sensible heat flux (upward).	&    ($W/m^2$)  \\
 lhf  &   147   &  1(T)	&surface latent heat flux (upward).	&    ($W/m^2$)  \\
 psi  &   148   &  3(U)	&stream function.		&    ($m^2/s$)      \\
 chi  &   149   &  3(U)	&velocity potential.		&    ($m^2/s$)      \\
 r    &   157   &  1(T)	&relative humidity.		&    (-)            \\
 uv10  &  159   &  1(T) &10 meter height wind magnitude.&    ($m/s$)        \\        
 runoffo& 160   &  1(T) &surface runoff over ocean.     &    ($m/s$)        \\    
 runoffl& 161   &  1(T) &surface runoff over land.      &    ($m/s$)        \\
 tcc    & 164   &  1(T) &total cloud cover.             &    (0-1)          \\
 palb   & 174   &  1(T)	&planetary albedo.              &    (\%)           \\
 alb   &  175   &  1(T)	&surface albedo.                &    (\%)           \\
 ssr   &  176   &  1(T)	&surface solar radiation (downward).&($W/m^2$)      \\
 tsr   &  178   &  1(T)	&top solar radiation (downward).    & ($W/m^2$)     \\
 str    & 177   &  1(T)	&surface thermal radiation (upward).& ($W/m^2$)     \\
 ttr    & 179   &  1(T)	&top thermal radiation (upward).    & ($W/m^2$)     \\
 ustress& 180   &  1(T) &U-stress.                           & ($Pa$)       \\
 vstress& 181   &  1(T) &V-stress.                           & ($Pa$)       \\
 evap  &  182   &  1(T)	&surface evaporation.		     & ($cm/year$)  \\
 pp    &  260   &  1(T)	&total precipitation (142+143).	     & ($cm/year$)  \\   
 hforc &  301   &  2(T)	&heating force.			     &  ($K/s$)     \\
 vforc &  302   &  3(U)	&potential vorticity forcing.	     &  ($1/s^2$)   \\
 ageu  &  303   &  3(U)	&ageostrophic wind in x direction.   &  ($m/s$)     \\
 agev  &  304   &  3(U)	&ageostrophic wind in y direction.   &  ($m/s$)     \\
 sp    &  134   &  1(T)	&surface pressure.                   &  ($Pa$)     \\
 eminp &  309   &  1(T)	&evaporation - precipitation (182-142-143).  & ($cm/year$) \\  \hline
\hline    
\end{tabular}
Talbe-1
\\
\noindent{4.2  Ocean Variables} 
\\ 
\\ 
The following variables can be requested for output from the ocean model (see Table-2).  
The code numbers here are self invented.
\\ 
\\
\\
\begin{tabular}{lllll}
name in & code   & level   &variable		  	        &    unit       \\
namelist  \\
\hline  
 to     &   401   &  12	&ocean temperature.	                &    ($^oC$)    \\
 sa     &   402   &  12	&ocean salinity.	                &    ($PSU$)    \\
 ro     &   403   &  12	&sea water density.	                &    ($kg/m^3$) \\
 hadvt  &   404   &  12	&horizontal (x+y) advection of temperature.& ($^oC/s$)  \\
 hadvs  &   405   &  12	&horizontal (x+y) advection of salinity.&    ($PSU/s$)  \\
 vadvt  &   406   &  12	&vertical advection of temperature.	&    ($^oC/s$)  \\
 vadvs  &   407   &  12	&vertical advection of salinity.	&    ($PSU/s$)  \\
 hdift  &   408   &  12	&horizontal (x+y) diffusion of temperature.& ($^oC/s$)  \\
 hdifs  &   409   &  12	&horizontal (x+y) diffusion of salinity.&    ($PSU/s$)  \\
 vdift  &   410   &  12	&vertical diffusion of temperature.	&    ($^oC/s$)  \\
 vdifs  &   411   &  12	&vertical diffusion of salinity.	&    ($PSU/s$)  \\
 convt  &   412   &  12	&convective adjustment of temperature.	&    ($^o$C/s)  \\
 convs  &   413   &  12	&convective adjustment of salinity.	&    ($PSU/s$)  \\
 relaxt &   415   &  1  &ocean heat forcing.                    &    ($^o$C/s)  \\
 relaxs &   416   &  1  &ocean salinity forcing.                &    ($PSU/s$)  \\
 uo     &   417   &  12	&x velocity of the ocean circulation.	&    ($m/s$)    \\
 vo     &   418   &  12	&y velocity of the ocean circulation.	&    ($m/s$)    \\
 wo     &   414   &  11	&z velocity of the ocean circulation.	&    ($m/s$)    \\
 heex   &   421   &  1  &heat flux from atmosphere to ocean.    &    ($W/m^2$)  \\
        &         &     &no sea-ice present: 176-146-147-177      &             \\
        &         &     &sea-ice present: 426                    &              \\
 saex  &    422   &  1  &fresh-water flux from atmosphere to ocean.   &($m/s$)  \\
       &        &       &no sea-ice present: 182+425-142-143-160 &              \\
       &        &       &sea-ice present: 425 - 160              &              \\
 brin  &  425  &   1    &fresh water flux from atmosphere to     &              \\
       &       &        &ocean due to ice changes.               &    ($m/s$)   \\       
 heico &  426  &    1   &heat flux from ice to ocean.            &    ($W/m^2$) \\
 dice  &  427  &    1   &ice thickness.                          &    ($m$)     \\
 sds   &  428  &    1   &snow depth over sea-ice.                &    ($m$)     \\ 
 tice  &  430  &    1   &temperature of ice.                     &    ($K$)     \\  \hline
\hline
\end{tabular}
Table-2


\newpage
\baselineskip=20pt
\section{Post Processing}

\noindent{5.1   DATA PROCESSING AND VISUALIZATION}
\\

In this section a description of procedures to process the output
dataset from the climate model is given.  

For each type of atmospheric data (as defined by the parameter meantype) and ocean data 
(as defined by mtype) the model writes all the variables that are requested by the user in one dataset.
The format is based on the GRIB format. Each entry consists of
a header and a field. 
There are 8 integers in the header. They are code number, level,
year, month, day, x dimension, y dimension, field identifier. The
last identifies the type of data: 0 for instantaneous fields, 1 for mean fields and 2 for
standard deviation fields. The dataset is in global grid points and unformatted.
 
To visualize these output data we need to convert this format into the format which can be read by
the graphic program GrADS. This is done by the program tograds\_at for atmospheric output.
The other thing this program does is to split the output data file into 2 grads data files with
one for T-level variables, one for U-level variables.
At the same time it writes control 
files for the two grads data files. The names of the grads data files
that you get after running this program are the original output data file's name 
with an extension grdstl.dat, grdsul.dat which corresponds to 
T-level and U-level variables.  One level variables are included in T-level variables with pressure 
level of 1000 hPa with no regards of the actual level in 
the model. Two-level variables which are defined on the upper two T-levels
are also included in T-level variables with constant zero fields on 1000 hPa level.  The names of the control
files have the extension grdstl.ctl, grdsul.ctl respectively. The reason of splitting the output dataset into
two grads datafiles is that in GrADS it is impossible to define pressure level for each
individual variable. In this atmospheric model however some  variables
are defined on T-level, some on U-level.
\newpage
In order to make it easier for
later visualization of the data it is important to define all the levels 
correctly. 

The program tograds\_at is stored in directory /usr/local/xbin on the PowerIndigo (bgwdm1). What
you need to do to be able to use this program is, first, to define the path /usr/local/xbin in your 
.login file. After this, in the directory where your data file exists
type:

tograds\_at outputdataname 
\\
Then tograds\_at will see outputdataname as the name of the dataset.
\\
For the ocean output the program tograds\_oc should be used. This program is stored in /usr/local/xbin as 
well. This program has the same function as tograds\_at for atmospheric data. It splits the data 
into two grads datasets. One for one level variables and the other for 12 level variables. 
For simplicity the vertical velocity of the ocean circulation which is defined on 11 levels is also
included in 12 level variables. The names of the grads data files and the grads control files have 
the same structure as atmospheric dataset, namely for grads data files the names are the
datafile's name with an extension of grds1.dat and grds12.dat for one level and twelve level variables 
respectively. For control files are the datafile's name with an extension of grds1.ctl and grds12.ctl
respectively. 

The usage of this program is the same as for atmospheric output.
In the directory where output dataset is stored type:
tograds\_oc outputdataname \\
Then tograds\_oc program will see outputdataname as the input data name.
\\
\\
\noindent{5.2   POST PROCESSING PROGRAM}
\\
\\
There are some post processing programs which are written by Koos Verbeek stored in the directory
~wang/lib on the bgwdm9. That directory also contains read.me file and a manual about the usage of these 
programs with filename manual.tex. \\
\newpage
\noindent{5.3   GRADS SCRIPTS}\\  \\
\noindent{5.3.1  Grads Graphic Scripts}\\
\\
There are a few grads graphic scripts available on the PowerIndigo in the
directory wang/GRADS\_LIB/version3.0. You can run these scripts in your own
working directory if you define the environment variable 'GASCRP' in your .login file.
For instance you can add the following line in your .login file: \\
setenv GASCRP $\simeq$wang/GRADS\_LIB/version3.0. \\
These scripts are meant to serve as example scripts. There are comments in 
each of the script. Users can easily modify them to serve their own needs. 
These scripts make multiple
plots depending on the time steps that you choose. In the Appendix a more
detailed information about each script is given.
These scripts are best suited for batch mode working. There are a few unix script jobs that
run these grads scripts stored in the same directory. In the Appendix the functions 
of these script jobs are also listed.
\\
\\
WARNING: to use these grads scripts interactively the PORTRAIT mode of GrADS session should be used.
\\
\\
\\
\noindent Overview \\
\\ 
\begin{tabular}{ll}

p1lev.sc1  &	plot 1 level variable in lat-lon projection      \\
p1lev.sc2  &	plot 1 level variable in north polar projection   \\
\\
p2lev.sc1  &	plot 2 level variable in lat-lon projection      \\
p2lev.sc2  &	plot 2 level variable in north polar projection   \\
\\
p3lev.sc1  &	plot 3 level variable in lat-lon projection      \\
p3lev.sc2  &	plot 3 level variable in north polar projection   \\
\\
pnlev.sc3  &	plot n level variable in lat-height section       \\
\\
\end{tabular}
\\
\begin{tabular}{ll}

Script Job name	&	Function  \\
\\
grads.job1      &	run p1lev.sc1      \\
grads.job2      &	run p1lev.sc2      \\
\\
grads.job3      &	run p2lev.sc1      \\
grads.job4      &	run p2lev.sc2      \\
\\
grads.job5      &	run p3lev.sc1      \\
grads.job6      &	run p3lev.sc2      \\
\\
grads.job7      &	run pnlev.sc3      \\
\\
\end{tabular}
\\
\\
\noindent{5.3.2  Grads Manipulation Scripts}\\
\\
There is a grads script which can be used to compute the depth-mean values for
different variables of the ocean model in the grads. This script is stored in directory ~wang/GRADS\_LIB
/version3.0 on the PowerIndigo. The values of the thickness of each level is as follows:
30, 50, 80, 140, 250, 350, 400, 450, 500, 550, 600, 600. The unit is in
meters. If your model has different level thickness than these the script should be modified.
In the Appendix a more detailed information about this script is given.
\\
\\
\\
\begin{tabular}{ll}

Script Name	&	Function  \\
\\
dm.sc           &   compute depth-mean in grads program  
\\
\end{tabular}
\\
\baselineskip=18pt
\\
\\
\newpage
\section{Appendix}
A: GrADS Scripts
\\
\noindent 1. p1lev.sc1  \\  \\
\begin{tabular}{ll}
Purpose:    &     plot 1 level variables, two time steps on one page.       \\
\\
Projection: &	  lat-lon projection                                        \\
\\
Use:        &     grads: run p1lev.sc1 ctlfilename varname t1 t2 string     \\
\\
Mode:       &     interactive and batch                                     \\
\\
Hardcopy:   &     varname.plot.'n'                                          \\                                        
*****  \\
ctlfilename: &   grads control file name            \\
varname:     &   variable name              \\       
t1:       &   begin time step          \\
t2:       &   end time step         \\
string:   &   user defined character string  \\
\end{tabular}
\\
** see figure.1  
\\
\\
\noindent 2. p1lev.sc2 \\ \\
\begin{tabular}{ll}
Purpose:    &   plot 1 level variables, two time steps on one page.         \\
\\
Projection: &   north polar projection                                      \\
\\
Use:	    &   grads: run p1lev.sc2 ctlfilename varname t1 t2 string       \\
\\
Mode:       &   interactive and batch                                       \\
\\
Hardcopy:   &   varname.plot.'n'                                            \\
*****  \\
ctlfilename: &  grads control file name                  \\
varname:     &  variable name                    \\
t1:       &  begin time step     \\
t2:       &  end time step        \\
string:   &   user defined character string  
\end{tabular}
\\
** see figure.2  
\\ 
\\
\newpage
\noindent 3. p2lev.sc1 \\ \\
\begin{tabular}{ll}
Purpose:     &  plot 2 level variables, two plots fill one page, one plot for one level.\\
\\
Projection:  &  lat-lon projection                                                     \\
\\
Use:         &  grads: run p2lev.sc1 ctlfilename varname t1 t2  string                 \\
\\
Mode:        &  interactive and batch                                                  \\
\\
Hardcopy:    &  varname.plot.'n'                                                       \\
*****  \\
ctlfilename: &  grads control file name    \\ 
varname:     &  variable name      \\ 
t1:       &  begin time step    \\
t2:       &  end time step  \\
string:   &   user defined character string  
\end{tabular}
** see figure.3
\\
\\ 
\noindent 4. p2lev.sc2 \\ \\
\begin{tabular}{ll}
Purpose:     &  plot 2 level variables, two plots fill one page, one plot for one level.\\
\\
Projection:  &  north polar projection                                                  \\
\\
Use:         &  grads: run p2lev.sc2 ctlfilename varname t1 t2  string                  \\
\\
Mode:        &  interactive and batch                                                   \\
\\
Hardcopy:    &  varname.plot.'n'                                                        \\
*****  \\ 
ctlfilename: &  grads control file name     \\ 
varname:     &  variable name       \\
t1:       &  begin time step     \\
t2:       &  end time step  \\
string:   &   user defined character string 
\end{tabular}
** see figure.4
\\ 
\\
\newpage
\noindent 5. p3lev.sc1 \\ \\
\begin{tabular}{ll}
Purpose:     &  plot 3 level variables, three plots fill one page, one plot for one level.\\
\\
Projection:  &  lat-lon projection                                                    \\
\\
Use:         &  grads: run p3lev.sc1 ctlfilename varname t1 t2 string                 \\
\\
Mode:        &  interactive and batch                                                 \\
\\
Hardcopy:    &  varname.plot.'n'                                                      \\
*****  \\
ctlfilename: &  grads control file name   \\
varname:     &  variable name       \\
t1:       &  begin time step    \\ 
t2:       &  end time step  \\
string:   &   user defined character string  
\end{tabular}
** see figure.5
\\ 
\\
\noindent 6. p3lev.sc2 \\ \\
\begin{tabular}{ll}
Purpose:     &  plot 3 level variables, three plots fill one page, one plot for one level.\\
\\
Projection:  &  north polar projection                                                \\
\\
Use:         &  grads: run p3lev.sc2 ctlfilename varname t1 t2  string                \\
\\
Mode:        &  interactive and batch                                                 \\
\\
Hardcopy:    &  varname.plot.'n'                                                      \\
*****  \\
ctlfilename: &  grads control file name \\
varname:     &  variable name           \\
t1:       &  begin time step            \\
t2:       &  end time step              \\
string:   &   user defined character string  
\end{tabular}
** see figure.6
\\ 
\\
\newpage
\noindent 7. pnlev.sc3 \\ \\
\begin{tabular}{ll}
Purpose:     &  plot n level variables, four plots fill one page, one plot for one time step.\\
\\
Projection:  &  lat-height projection                                                  \\
\\
Use:         &  grads: run pnlev.sc3 ctlfilename varname lev1 lev2 t1 t2 t3 t4 string  \\
\\
Mode:        &  interactive and batch                                                  \\
\\
Hardcopy:    &  varname.plotv.'n'                                                      \\
*****  \\
ctlfilename: &  grads control file name  \\                                    
varname:     &  variable name            \\               
lev1:     &  lower pressure level        \\
lev2:     &  upper pressure level        \\
t1:       &  first time step             \\  
t2:       &  second time step            \\
t3:       &  third time step             \\  
t4:       &  fourth time step            \\
string:   &  user defined character string  
\end{tabular}
\\
** see figure.7
\\
\\
\\

\noindent 8. dm.sc \\ \\
\begin{tabular}{ll}
Purpose:     &  compute depth-mean values for various variables.   \\
\\
Projection:  &  none                                               \\
\\
Use:         &  grads: run dm.sc ctlfilename varname               \\
\\
Mode:        &  interactive                                        \\
\\
Hardcopy:    &  none                                               \\
*****  \\
ctlfilename: &  grads control file name  \\                                    
varname:     &  variable name    \\               
\end{tabular}
\\
\newpage
B: Job Scripts
\\
\\
\\
\noindent 1. grads.job1 \\ \\
\begin{tabular}{ll}
Purpose:     &  run p1lev.sc1, transfer produced plot(s) from gx format to ps, send \\
             &  plot(s) directly to printer  \\
\\
Use:         &  unix prompt: grads.job1 ctlfilename varname t1 t2 dir  string    \\
Mode:        &  interactive and batch        \\ 
\\
*****  \\
ctlfilename: &  grads control file name     \\
varname:     &  variable name     \\
t1:       &  begin time step    \\
t2:       &  end time step   \\ 
dir:      &  directory name where data/control file stored \\ 
string:   &  user defined character string  
\end{tabular}
\\
\\
\\
\\ 
\noindent 2. grads.job2 \\ \\
\begin{tabular}{ll}
Purpose:     &  run p1lev.sc2, transfer produced plot(s) from gx format to ps, send \\
             &  plot(s) directly to printer  \\
\\
Use:         &  unix prompt: grads.job2 ctlfilename varname t1 t2 dir  string     \\
Mode:        &  interactive and batch        \\ 
\\
*****  \\
ctlfilename: &  grads control file name     \\
varname:     &  variable name     \\
t1:       &  begin time step    \\
t2:       &  end time step    \\
dir:      &  directory name where data/control file stored \\ 
string:   &  user defined character string  
\end{tabular}
\\ 
\newpage
\noindent 3. grads.job3 \\ \\
\begin{tabular}{ll}
Purpose:     &  run p2lev.sc1, transfer produced plot(s) from gx format to ps, send \\
             &  plot(s) directly to printer  \\
\\
Use:         &  unix prompt: grads.job3 ctlfilename varname t1 t2 dir  string     \\
Mode:        &  interactive and batch        \\ 
\\
*****  \\
ctlfilename: &  grads control file name     \\
varname:     &  variable name     \\
t1:       &  begin time step    \\
t2:       &  end time step    \\
dir:      &  directory name where data/control file stored  \\
string:   &  user defined character string  
\end{tabular}
\\
\\
\\
\\ 
\noindent 4. grads.job4 \\ \\
\begin{tabular}{ll}
Purpose:     &  run p2lev.sc2, transfer produced plot(s) from gx format to ps, send \\
             &  plot(s) directly to printer  \\
\\
Use:         &  unix prompt: grads.job4 ctlfilename varname t1 t2 dir  string     \\
Mode:        &  interactive and batch        \\ 
\\
*****  \\
ctlfilename: &  grads control file name     \\
varname:     &  variable name     \\
t1:       &  begin time step    \\
t2:       &  end time step    \\
dir:      &  directory name where data/control file stored  \\
string:   &  user defined character string  
\end{tabular}
\\
\newpage 
\noindent 5. grads.job5 \\ \\
\begin{tabular}{ll}
Purpose:     &  run p3lev.sc1, transfer produced plot(s) from gx format to ps, send \\
             &  plot(s) directly to printer  \\
\\
Use:         &  unix prompt: grads.job5 ctlfilename varname t1 t2 dir  string     \\
Mode:        &  interactive and batch        \\ 
\\
*****  \\
ctlfilename: &  grads control file name     \\
varname:     &  variable name     \\
t1:       &  begin time step    \\
t2:       &  end time step    \\
dir:      &  directory name where data/control file stored  \\
string:   &  user defined character string  
\end{tabular}
\\
\\
\\
\\
\noindent 6. grads.job6 \\ \\
\begin{tabular}{ll}
Purpose:     &  run p3lev.sc2, transfer produced plot(s) from gx format to ps, send \\
             &  plot(s) directly to printer  \\
\\
Use:         &  unix prompt: grads.job6 ctlfilename varname t1 t2 dir  string     \\
Mode:        &  interactive and batch        \\ 
\\
*****  \\
ctlfilename: &  grads control file name     \\
varname:     &  variable name     \\
t1:       &  begin time step    \\
t2:       &  end time step    \\
dir:      &  directory name where data/control file stored   \\
string:   &  user defined character string 
\end{tabular}
\\
\newpage 
\noindent 7. grads.job7 \\ \\
\begin{tabular}{ll}
Purpose:     &  run pnlev.sc3, transfer produced plot(s) from gx format to ps, send \\
             &  plot(s) directly to printer  \\
\\
Use:         &  unix prompt: grads.job7 ctlfilename varname lev1 lev2 t1 t2 t3 t4 dir string\\
Mode:        &  interactive and batch        \\ 
\\
*****  \\
ctlfilename: &  grads control file name     \\
varname:     &  variable name      \\
lev1      &  lower level pressure  \\
lev2      &  upper level pressure  \\
t1:       &  first time step       \\
t2:       &  second time step      \\
t3:       &  third time step       \\
t4:       &  fourth time step       \\
dir:      &  directory name where data/control file stored  \\ 
string:   &  user defined character string 
\end{tabular}
\\ 


\newpage
Figure.1\\
\\
\\
Figure.2\\
\\
\\
Figure.3\\
\\
\\
Figure.4\\
\\
\\
Figure.5\\
\\
\\
Figure.6\\
\\
\\
Figure.7\\


\end{document}

