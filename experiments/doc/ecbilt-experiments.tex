\documentclass[a4paper,11pt]{article}

\usepackage[margin=1in]{geometry}
\usepackage[utf8]{inputenc}
\usepackage{fourier}
\usepackage{amsmath,amssymb}

\usepackage{color}
\definecolor{orange}{rgb}{0.75,0.5,0}
\definecolor{magenta}{rgb}{1,0,1}
\definecolor{cyan}{rgb}{0,1,1}
\definecolor{grey}{rgb}{0.25,0.25,0.25}
\newcommand{\outline}[1]{{\color{grey}{\scriptsize #1}}}
\newcommand{\revnote}[1]{{\color{red}\textit{\textbf{#1}}}}
\newcommand{\note}[1]{{\color{blue}\textit{\textbf{#1}}}}
\newcommand{\citenote}[1]{{\color{orange}{[\textit{\textbf{#1}}]}}}

\usepackage{xspace}
\newcommand{\zfive}{\ensuremath{Z_{500}}\xspace}

\usepackage{listings}
\usepackage{courier}
\lstset{basicstyle=\tiny\ttfamily,breaklines=true,language=Haskell}

\usepackage{fancyvrb}
\usepackage{url}

\usepackage{tikz}
\usetikzlibrary{arrows,positioning}

\usepackage{natbib}

\DeclareMathOperator{\diag}{diag}
\DeclareMathOperator*{\argmax}{arg \, max}

\usepackage{subfig}
\usepackage{float}
\newfloat{listing}{tbp}{lop}
\floatname{listing}{Listing}

\title{Standalone ECBILT Experiments}
\author{Ian~Ross}
\date{13 January 2015 -- ???}

\begin{document}

\maketitle

%======================================================================
\section{Introduction}

The idea here is to do some experiments with a standalone version of
the ECBILT 3-level atmosphere model to decide whether it's a suitable
candidate for coupling into GENIE.  The two main things to consider
here are the speed of the ECBILT model and how good its climatology
is.  To investigate these things, I've done a quick first comparison
with a HadAM3 simulation (BRIDGE simulation \texttt{tcszd}).  I've
also done some investigational work to help think about how hard it
would be to couple ECBILT to GENIE, and to look at potential problems
with the ECBILT model parameterisations for likely GENIE applications.

There are two versions of ECBILT that could be used for this
comparison.  The first is already coupled into the LOVECLIM EMIC, so
would require some work to decouple it and set it up for standalone
operation.  The second is the original standalone ECBILT
model\footnote{Downloaded from here
  \url{http://knmi.nl/~selten/pro_ecbilt.html}.}, which is what I'm
going to start with.  It has a couple of deficiencies compared to the
version included in LOVECLIM, but it's good enough for a quick first
look.  The main deficiency of the standalone ECBILT version is that
it's not possible to change GHG forcings.  This is because of the way
the LW radiation scheme works: it's an empirical fit to LW radiation
absorption from the GFDL GCM.  I'm not yet sure what GHG conditions
that fitting was done against, but I'm going to press on assuming that
it was pre-industrial, as for the UM simulation.  For applications in
GENIE, we obviously need to be able to change the GHG concentrations
so, if results from the first experiments with the standalone ECBILT
are promising, I'll probably just transplant the newer LW radiation
scheme from the LOVECLIM version of ECBILT (which treats GHGs
explicitly) into the standalone ECBILT.

The approach I'm going to take to model validation here is just to
convert boundary conditions from a suitable HadAM3 run to the form
required to drive ECBILT, do the same sort of atmosphere-only
simulation as done in the HadAM3 job and to compare the parts of the
climatology most important for long-term GENIE simulations between the
two models (also looking at model performance along the way).

As a first simple attempt, I'm going to leave all ECBILT input files
unchanged from their defaults except for SST and sea-ice forcing,
which are adapted from the HadAM3 simulation.  This approach means
that there's no need to convert land/sea masks, land cover types, the
lake mask used by ECBILT, and so on.  If we decide to go further, I'll
do simulations using a land/sea mask and other boundary conditions
based on the HadAM3 data (probably also updating the ECBILT LW
radiation scheme as mentioned above).


%======================================================================
\section{Code and test platform setup}

All the experiments reported here are performed on a machine running
Arch Linux (rolling release as of 18 January 2015), with a 3\,GHz
Intel Core i5-3330 CPU and 16\,Gb of main memory.  All test runs were
performed on a lightly loaded machine.  Very few code changes were
needed to get the standalone ECBILT code to build with the compiler
used (GFortran 4.9.2).


%======================================================================
\section{Experiment \#1}

\begin{itemize}
  \item{Run name: \texttt{expt-1}.}
  \item{Original standalone ECBILT code, with only minimal changes for
    compilation.}
  \item{SST and sea-ice forcing converted from UM \texttt{tcszd}
    experiment.}
  \item{Simulation length: 100 years; spin-up: 70 years; analysis: 30
    years.}
\end{itemize}

%----------------------------------------------------------------------
\subsection{Boundary conditions}

Boundary conditions to run the standalone ECBILT model were derived
from the BRIDGE \texttt{tcszd} HadAM3 simulation.  This is a 100-year
pre-industrial atmosphere-only simulation with prescribed
climatological SSTs and sea ice.

In order to get an initial ECBILT simulation set up quickly, I decided
to convert only the SST and sea ice forcing from the HadAM3
simulation, keeping the original ECBILT land/sea mask, orography, lake
mask and land surface conditions (e.g. albedo), GHG concentrations and
so on.  Modifying these latter conditions to match HadAM3 will take
more work than the relatively simple SST and sea ice regridding.  I'll
go ahead and do that if the results from the first simulation are
encouraging enough.

\begin{figure}
  \begin{center}
    \includegraphics[width=\textwidth]{../fixed-data/plots/orog-plots}
  \end{center}
  \caption{ECBILT and UM orography and land/sea masks used for
    comparison experiments.}
  \label{fig:orog}
\end{figure}

Figure~\ref{fig:orog} shows the land/sea mask and orography for ECBILT
and HadAM3 for comparison.  One comment needs to be made about the
ECBILT land/sea mask here: although some areas that should be ocean
are treated as land in the land/sea mask, in particular the
Mediterranean, the Great Lakes and the South China Sea, these grid
cells \emph{are} treated appropriately as water areas, via the use of
a seperate lake mask identifying grid cells that have a significant
water fraction.

\subsubsection{SST and sea-ice}

Conversion of SST and sea-ice from the HadAM3 boundary conditions to
the form needed by ECBILT is pretty straightforward.

For the SST, the HadAM3 boundary conditions are given as a monthly
climatology of SST.  ECBILT requires daily SST on a different grid, so
I take the HadAM3 monthly SST fields, Poisson fill to get rid of
missing values, regrid to the ECBILT grid, mask with the ECBILT
land/sea mask, then use independent periodic interpolation in time at
each grid point to generate a daily SST climatology.  This data is
then converted to the binary format read by ECBILT.

For sea ice, again HadAM3 again uses a monthly climatology, but ECBILT
uses a ``birth/death'' map showing, for each grid cell, the month of
the year where sea ice first appears and the month when it disappears.
To generate a sea ice driver file in this form, the HadAM3 monthly sea
ice data is first Poisson filled, regridded and masked to generate a
monthly sea ice climatology on the ECBILT grid.  Next, some simple
heuristics are used to determine ``birth'' and ``death'' months for
sea ice in each grid cell, and this information is encoded into the
ASCII file format used by ECBILT.

%----------------------------------------------------------------------
\subsection{Model performance}

The 100-year simulation performed here took about 75 minutes
(wallclock and CPU time were about the same), equating to
\textbf{about 45 seconds per year of simulation}.  Replacing the
existing LW radiation scheme in ECBILT with the more explicit scheme
in the LOVECLIM code will probably slow this down a little bit, but
not much.

The ECBILT code is all serial, so there may be opportunities for
parallelisation if performance is a problem.  The atmosphere runs at a
longitude/latitude resolution of $64 \times 32$ using a spectral
method for stepping the dynamics.  I've not yet done any profiling to
determine where the best places to look for optimisations are, but
that would be the obvious thing to do.

%----------------------------------------------------------------------
\subsection{Model climatology}

Figures~\ref{fig:ts-1}--\ref{fig:stress-1} show seasonal climatologies
of surface temperature, precipitation, evaporation, $P-E$ and
atmospheric circulation for the ECBILT Experiment \#1 simulation --
these should be compared with the HadAM3 HadAM3 \texttt{tcszd}
simulation results in
Figures~\ref{fig:ts-hadam3}--\ref{fig:stress-hadam3} in
Appendix~\ref{sec:hadam3}.

\paragraph{Surface temperature}

The large-scale patterns of surface temperature field in the ECBILT
simulation are pretty good.  Because of the SST forcing, it's a
relatively easy field to get right, but it does show that there are no
gross problems with the model.  What differences there are between the
ECBILT and HadAM3 fields appear mostly to be due to the rougher
orography in ECBILT -- the Tibetan plateau and western North America
near the Rockies are the areas where this is most obvious.

\begin{figure}
  \begin{center}
    \includegraphics[width=\textwidth]{../expt-1/plots/ts-plots}
  \end{center}
  \caption{Seasonal surface temperature for ECBILT Experiment \#1.}
  \label{fig:ts-1}
\end{figure}

\paragraph{Precipitation}

The differences between the ECBILT and HadAM3 fields here are more or
less what you would expect from a comparison between a lower
resolution (in both the horizontal but also, more importantly, in the
vertical direction) and a higher resolution model.  Winter storm
tracks in the Atlantic are not well represented in ECBILT and tropical
convective precipitation is much less well localised in ECBILT, with a
large diffuse region of precipitation stretching from the western into
the central Pacific.  HadAM3 shows tropical precipitation much more
localised over the western Pacific warm pool and narrow bands either
side of the equator across the rest of the Pacific.  These kinds of
deficiencies in precipitation modelling are kind of inevitable in a
model with only three levels in the vertical.

\begin{figure}
  \begin{center}
    \includegraphics[width=\textwidth]{../expt-1/plots/pp-plots}
  \end{center}
  \caption{Seasonal precipitation for ECBILT Experiment \#1.}
  \label{fig:pp-1}
\end{figure}

\paragraph{Evaporation}

Similar comments apply here as to the precipitation comparison:
evaporation from the ocean is much less well spatially localised in
ECBILT, with a large region of the western Pacific showing high
evaporation values and no clear ITCZ being visible.  Again, hard to
get this right even in a model with more levels in the vertical, let
alone one with only three!

\begin{figure}
  \begin{center}
    \includegraphics[width=\textwidth]{../expt-1/plots/evap-plots}
  \end{center}
  \caption{Seasonal evaporation for ECBILT Experiment \#1.}
  \label{fig:evap-1}
\end{figure}

\paragraph{$P-E$}

The spatial patterns of $P-E$ in ECBILT exhibit the same problems seen
in the individual precipitation and evaporation plots: the
hydrological cycle is generally much more spatially diffuse than in
the HadAM3 simulation (and in reality).  For both models, the mean
annual hydrological cycle is more or less balanced (for HadAM3 the
area-averaged annual imbalance is about $-2$\,mm and for ECBILT it's
about 23\,mm).  The patterns of $P-E$ in ECBILT are weaker and more
diffuse than HadAM3, although if you squint a little, the spatial
patterns of positive and negative moisture budget are more or less
right.

\begin{figure}
  \begin{center}
    \includegraphics[width=\textwidth]{../expt-1/plots/pmine-plots}
  \end{center}
  \caption{Seasonal precipitation minus evaporation for ECBILT
    Experiment \#1.}
  \label{fig:pmine-1}
\end{figure}

\paragraph{Winds}

Finally, the distribution of upper and lower level winds and vertical
pressure velocity follow more or less the pattern that you would
expect: the atmospheric circulation in ECBILT is weaker and more
diffuse than in HadAM3, particularly in the vertical direction.  This
vertical smoothing because of the small number of vertical levels in
ECBILT is almost certainly the source of the weaker hydrological cycle
seen in the $P-E$ plots.  Slightly surprisingly though, the spatial
patterns and magnitude of the wind stress look pretty good, especially
over the oceans.

\begin{figure}
  \begin{center}
    \includegraphics[width=\textwidth]{../expt-1/plots/wind-plots}
  \end{center}
  \caption{Seasonal circulation plots for ECBILT Experiment \#1:
    arrows show upper level and lower level winds, colours show middle
    atmosphere vertical pressure velocity.}
  \label{fig:wind-1}
\end{figure}


\begin{figure}
  \begin{center}
    \includegraphics[width=\textwidth]{../expt-1/plots/stress-plots}
  \end{center}
  \caption{Seasonal surface wind stress plots for ECBILT Experiment
    \#1.}
  \label{fig:stress-1}
\end{figure}


%======================================================================
\section{Experiment \#2}

\begin{itemize}
  \item{Run name: \texttt{expt-2}.}
  \item{Original standalone ECBILT code, with only minimal changes for
    compilation.}
  \item{Land/sea mask, orography, SST and sea-ice forcing converted
    from UM \texttt{tcszd} experiment.}
  \item{River routing mask converted to new land/sea mask by hand.}
  \item{Original ECBILT albedo data: the albedo code uses a simple
    interpolation scheme based on fixed latitude-dependent values read
    in at startup for bare-soil, TOA and ocean albedo.  This will need
    to be modified to calculate albedos based on land surface
    characteristics.}
  \item{Simulation length: 100 years; spin-up: 70 years; analysis: 30
    years.}
\end{itemize}

%----------------------------------------------------------------------
\subsection{Boundary conditions}

\subsubsection{Land/sea mask and orography}

For this experiment, as well as using the HadAM3 \texttt{tcszd} SST
and sea-ice forcing, the model land/sea mask, orography and river
routing were adapted based on the HadAM3 input data.
Figure~\ref{fig:orog-2} shows the ECBILT land/sea mask and orography
derived by regridding the HadAM3 orography with the original HadAM3
data shown alongside for comparison.

As well as a land/sea mask, ECBILT also makes use of a lake mask --
this simply masks areas of water that are not included in the land/sea
mask.  In this case, all land/sea contrasts are already included in
the input land/sea mask derived from the HadAM3 data, so a dummy ``do
nothing'' lake mask is generated by hand so as not to disrupt the
ECBILT input data processing.

\begin{figure}
  \begin{center}
    \includegraphics[width=\textwidth]{../expt-2/plots/orog-plots}
  \end{center}
  \caption{ECBILT and UM orography and land/sea masks used for
    Experiment \#2.}
  \label{fig:orog-2}
\end{figure}

\subsubsection{River routing}

Since the land/sea mask was changed from the ECBILT default for this
experiment, it was also necessary to change the river routing maps
used for diverting runoff on land to the coasts.
Figure~\ref{fig:river-routing} shows the original ECBILT river routing
map along with the new routing map based on the HadAM3 land/sea mask,
which was generated by hand-editing the original ECBILT routing map.

\begin{figure}
    \fontsize{5.5}{5.5}\selectfont
\begin{verbatim}
                                                         EE                                                                  EE
                    A                              ccceeeeeeeeeeeE                      A                              ccceeeeeeeeeeE
     FFFF   Aaaaaaaaaaaa     A             C       ccc  EeeeeeeeE          FF      A   aaaaa     A                     ccc  EeeeeeeeE
     FfffaaaaaaaaaaaaaaaaaaaaaaaaaaaA bbbbcccccccccccc   EeeeeeE          Fff      aaaaaaaaaaaaaaaaaA      bb   Ccccc ccccc  EeeeeeE
    Fffffaaaaaaaaaaaaaaaaaaaiiiiaaaa  bbbbbbcccccddD      EeeE          Fffffaaaaaaaaaaaaaaaaaaaiiiiaaa   bbbbbbcccccdddD  c EeeE   e
   Ffffffgggggaaaaaaaaaaaaaii  ii      bBBBbbbcccddDDDddmM E      F      Ff fgggggaaaaaaaaaaaaaii   i       BBBbbbcccddDDDdd  E
   ffffgggggggggaaaaaaaiiiiii  iI          BbbddddddddmmmmM      Ff    fFffgggggggggaaaaaaaiiiii   iI          Bbbddddddd mmmmM      F
   gggggggggggggggiiiiiiiiiiiIII            KkkllllllmmmmmM      Ff    gggggggggggggggiiiiiiiiiiiIII            Kkklllll mmmMmM      F
   gGGggGGGgghhhhhiiijjjjjjjJ              KKkklllllllmM         Ff    gGgGggGggghhhhhiiijjjjjj jJ              KkklllllllmM         Ff
   GGGGGGghhhhhhhhhhhjjjjjjJ                 KkkllllllM           G    ggGGGGghhhhhhhhhhhjjjjJJjjJ               KkkllllllM           g
   ggggggghhhhhhhhhhhjjjjjJ                   Kkllllll          Ggg    ggggggghhhhhhhhhhhjjjjJJJJ                 Kklllll             g
   nnggggghhhhhhhhhhhjjjjJ                    KkklLL           NNnn    nnggggghhhh hhhhhhjjjjJ                     KkllLL          NNnn
   ooooogghhhhH hhHHhjjJ                       KKkkl           Nnnn    ooooogghhhh HhhhHhjj                         Kkkl  l        Nnnn
   ooooogghhh    h   hhRjJ                       KKkl          Nnoo    ooooogghh     hH  hh RjR                      KKkl          Nnoo
   ooooogghhh    h   RRRjjJ                        Ttuuuu       Ooo    ooooogghhh         RRRjR                         Tuuuu       Ooo
    OoooggppP        rrrrrR                         TtuuuuuU            OoooggppP        rRrrR                           TuuuuU
    OooooppP         RrrrrRrrR                      Ttuuuuuuu           OooooppP         rRrrR  r                       Ttuuuuuu
    OooooppP           RrrRrrr                      Ttuuuuuuu            OoooppP           RRRRRRr                      Ttuuuuuuu
     OoppppP             RRRRR                       Ttuuuuuu            OoppppP              ss                         Ttuuuuuu
     QqqqpP              sssss                        Ttvvvv             QqqqpP p            sssss                        Ttvvvvv
     QqqqpP             sssssss                       Ttvvvv             QqqqpPp            sssssss                       Ttvvv
      Qqq               sssssss                       TtvvvV             Qqqq               sssssss                      TtvvvvV
                          SSss                       TtvvvV                                   SSss                       TtvvvV
                                                     Ttv                                             Xx                  Ttv
                                                     Ttv                                                                 Ttv
                                                      T                                                                   T

        WWWWWWWWWWWWWWWWWWWWWWWWW                      WW                    WWWWWWWWWWWWWWWWWWWWWWW                       WW
   WWWWWwwwwwwwwwwwwwwwwwwwwwwwwwW         WWWWWWWWWWWWwwW     WWWW    WWWWWWwwwwwwwwwwwwwwwwwwwwwwwW          WWWWWWWWWWWWwwW     WWWW
   wwwwwwwwwwwwwwwwwwwwwwwwwwwwwwW     WWWWwwwwwwwwwwwwwwW   WWwwww    wwwwwwwwwwwwwwwwwwwwwwwwwwwwwwWWWWWWWWWWwwwwwwwwwwwwwwWWWWWWwwww
   wwwwwwwwwwwwwwwwwwwwwwwwwwwwwwWWWWWWwwwwwwwwwwwwwwwwwwwWWWwwwwww    wwwwwwwwwwwwwwwwwwwwwwwwwwwwwwwwwwwwwwwwwwwwwwwwwwwwwwwwwwwwwwww
   wwwwwwwwwwwwwwwwwwwwwwwwwwwwwwwwwwwwwwwwwwwwwwwwwwwwwwwwwwwwwwww    wwwwwwwwwwwwwwwwwwwwwwwwwwwwwwwwwwwwwwwwwwwwwwwwwwwwwwwwwwwwwwww
\end{verbatim}
\caption{River routing maps for original ECBILT land/sea mask (left)
  and new mask derived from HadAM3 land/sea mask (right).  Lower case
  letters show land points composing each drainage basin, while upper
  case letters indicate the corresponding ocean points into which each
  basin drains.}
\label{fig:river-routing}
\end{figure}

\subsubsection{SST and sea-ice}

The HadAM3 \texttt{tcszd} SST and sea-ice forcing were regridded for
use with ECBILT in the same way as described for Experiment \#1,
although using the new ECBILT land/sea mask derived from the HadAM3
\texttt{tcszd} land/sea mask.

%----------------------------------------------------------------------
\subsection{Model performance}

The 100-year simulation performed here took about 78 minutes, equating
to \textbf{about 47 seconds per year of simulation}.  The differences
in the land/sea mask probably account for the small difference in
timing compared to Experiment \#1.

%----------------------------------------------------------------------
\subsection{Model climatology}

Figures~\ref{fig:ts-2}--\ref{fig:stress-2} show seasonal climatologies
of surface temperature, precipitation, evaporation, $P-E$ and
atmospheric circulation for the ECBILT Experiment \#2 simulation --
these should be compared with the HadAM3 HadAM3 \texttt{tcszd}
simulation results in
Figures~\ref{fig:ts-hadam3}--\ref{fig:stress-hadam3} in
Appendix~\ref{sec:hadam3} and the Experiment \#1 results in
Figures~\ref{fig:ts-1}--\ref{fig:stress-1}.

There are small differences in all of the fields between Experiments
\#1 and \#2, but none of these differences are very large, and all are
well within the range of variability one would expect from a
low-resolution model like ECBILT.  The model climatology thus does not
appear to be highly sensitive to the land/sea mask and river routing
definitions used, which makes it reasonable to think that ECBILT
should provide a reasonable climatology when run using modern-day
GENIE land/sea masks and river routing information.

\begin{figure}
  \begin{center}
    \includegraphics[width=\textwidth]{../expt-2/plots/ts-plots}
  \end{center}
  \caption{Seasonal surface temperature for ECBILT Experiment \#2.}
  \label{fig:ts-2}
\end{figure}

\begin{figure}
  \begin{center}
    \includegraphics[width=\textwidth]{../expt-2/plots/pp-plots}
  \end{center}
  \caption{Seasonal precipitation for ECBILT Experiment \#2.}
  \label{fig:pp-2}
\end{figure}

\begin{figure}
  \begin{center}
    \includegraphics[width=\textwidth]{../expt-2/plots/evap-plots}
  \end{center}
  \caption{Seasonal evaporation for ECBILT Experiment \#2.}
  \label{fig:evap-2}
\end{figure}

\begin{figure}
  \begin{center}
    \includegraphics[width=\textwidth]{../expt-2/plots/pmine-plots}
  \end{center}
  \caption{Seasonal precipitation minus evaporation for ECBILT
    Experiment \#2.}
  \label{fig:pmine-2}
\end{figure}

\begin{figure}
  \begin{center}
    \includegraphics[width=\textwidth]{../expt-2/plots/wind-plots}
  \end{center}
  \caption{Seasonal circulation plots for ECBILT Experiment \#2:
    arrows show upper level and lower level winds, colours show middle
    atmosphere vertical pressure velocity.}
  \label{fig:wind-2}
\end{figure}

\begin{figure}
  \begin{center}
    \includegraphics[width=\textwidth]{../expt-2/plots/stress-plots}
  \end{center}
  \caption{Seasonal surface wind stress plots for ECBILT Experiment
    \#2.}
  \label{fig:stress-2}
\end{figure}


%======================================================================
\section{Experiment \#3}

\begin{itemize}
  \item{Run name: \texttt{expt-3}.}
  \item{LOVECLIM ECBILT code adapted to run as standalone atmosphere
    model.}
  \item{Boundary conditions as for Experiment \#2.}
  \item{Greenhouse gas concentrations: fixed at 286.43 ppmv
    ($\mathrm{CO_2}$), 796.60 ppbv (methane), 275.40 ppbv
    ($\mathrm{N_2O}$); 25.0 DU (tropospheric ozone).}
  \item{Simulation length: 100 years; spin-up: 70 years; analysis: 30
    years.}
\end{itemize}

%----------------------------------------------------------------------
\subsection{Code changes}

Here, I took the ECBILT code as used in LOVECLIM and disentangled it
from the LOVECLIM coupling code to make a standalone atmospheric
model, replacing the CLIO ocean GCM from LOVECLIM with the fixed ocean
model from the original ECBILT and replacing the VECODE dynamic
vegetation model with a fixed land surface scheme.

To do this, I vastly simplified the original LOVECLIM model
configuration and build process to have a standalone ``Fortran +
makefiles'' setup for building ECBILT.  I stubbed out all of the
LOVECLIM-related coupling code, then added back the fixed ocean and
land surface from the original ECBILT bit-by-bit.  After getting all
this to build, I did a bit of tidying up then reorganised all the
boundary condition file handling to make setting up experiments
simpler.

The standalone atmosphere model didn't work right away.  There were a
couple of problems that meant I had to make some model changes, and
they both required quite a bit of time to track down:

\begin{enumerate}
  \item{I had to fix the calculation of horizontal moisture diffusion,
    which had changed from the original ECBILT code and led to all
    sorts of numerical problems and completely unrealistic looking
    moisture fields.  I have no idea at all how this ever worked in
    LOVECLIM since the equation for the diffusion was missing some
    essential scaling constant.  Perhaps the moisture field was
    corrected somewhere else in the coupled model?  I could never
    track down what was happening, but reverting to the moisture
    diffusion calculation from the original ECBILT made the problems
    go away.}
  \item{I have to clamp sea ice temperature to a minimum value of
    200K\,K.  The LOVECLIM ECBILT radiation and atmospheric
    temperature profile calculations are based on interpolation from
    reference atmospheric profiles.  Some of these reference profiles
    have surface temperature inversions (i.e. the lowest level in the
    atmospheric profile is colder than higher layers), which can mess
    up the interpolation of temperature profiles down to the surface,
    producing unrealistically cold conditions in some places.  The
    best solution to this seems just to be to clamp the temperature
    and prevent these cold excursions.}
\end{enumerate}

%----------------------------------------------------------------------
\subsection{Boundary conditions}

The boundary conditions used for this experiment are essentially
identical to those for Experiment \#2.

%----------------------------------------------------------------------
\subsection{Model performance}

The 100-year simulation performed here took about 58 minutes, equating
to \textbf{about 35 seconds per year of simulation}.  I'm not sure why
the timing here is so different from Experiment \#2.

%----------------------------------------------------------------------
\subsection{Model climatology}

Figures~\ref{fig:ts-3}--\ref{fig:stress-3} show seasonal climatologies
of surface temperature, precipitation, evaporation, $P-E$ and
atmospheric circulation for the ECBILT Experiment \#3 simulation --
these should be compared with the HadAM3 HadAM3 \texttt{tcszd}
simulation results in
Figures~\ref{fig:ts-hadam3}--\ref{fig:stress-hadam3} in
Appendix~\ref{sec:hadam3} and the Experiment \#2 results in
Figures~\ref{fig:ts-2}--\ref{fig:stress-2}.

These initial results don't look all that great -- the following
problems can be identified immediately:
\begin{itemize}
  \item{The polar regions seem to be very cold.  This may be related
    to the problems with reference atmospheric profiles described
    above.}
  \item{There seems to be something wrong with the land surface
    scheme, since there's no evaporation occurring over the land.}
  \item{The winds are generally too weak, and wind stress in the
    Southern Ocean is very weak indeed in the southern summer.}
\end{itemize}

That said, the $P-E$ pattern looks better than for the other ECBILT
simulations, and the atmospheric circulation really isn't all that bad
for a first set of simulations with a new model setup.

\begin{figure}
  \begin{center}
    \includegraphics[width=\textwidth]{../expt-3/plots/ts-plots}
  \end{center}
  \caption{Seasonal surface temperature for ECBILT Experiment \#3.}
  \label{fig:ts-3}
\end{figure}

\begin{figure}
  \begin{center}
    \includegraphics[width=\textwidth]{../expt-3/plots/pp-plots}
  \end{center}
  \caption{Seasonal precipitation for ECBILT Experiment \#3.}
  \label{fig:pp-3}
\end{figure}

\begin{figure}
  \begin{center}
    \includegraphics[width=\textwidth]{../expt-3/plots/evap-plots}
  \end{center}
  \caption{Seasonal evaporation for ECBILT Experiment \#3.}
  \label{fig:evap-3}
\end{figure}

\begin{figure}
  \begin{center}
    \includegraphics[width=\textwidth]{../expt-3/plots/pmine-plots}
  \end{center}
  \caption{Seasonal precipitation minus evaporation for ECBILT
    Experiment \#3.}
  \label{fig:pmine-3}
\end{figure}

\begin{figure}
  \begin{center}
    \includegraphics[width=\textwidth]{../expt-3/plots/wind-plots}
  \end{center}
  \caption{Seasonal circulation plots for ECBILT Experiment \#3:
    arrows show upper level and lower level winds, colours show middle
    atmosphere vertical pressure velocity.}
  \label{fig:wind-3}
\end{figure}

\begin{figure}
  \begin{center}
    \includegraphics[width=\textwidth]{../expt-3/plots/stress-plots}
  \end{center}
  \caption{Seasonal surface wind stress plots for ECBILT Experiment
    \#3.}
  \label{fig:stress-3}
\end{figure}

%======================================================================
\section{ECBILT/GENIE compatibility}

%----------------------------------------------------------------------
\subsection{Code compatibility}

The notes here are based on a comparison between the ECBILT code and
the existing EMBM atmosphere model within GENIE.

\paragraph{Model initialisation}

This should be straightforward.  Both EMBM and ECBILT read a range of
model parameters from files during initialisation (as well as initial
atmospheric state), and only a little reorganisation of the ECBILT
code would be required.  One more significant change that it might be
worth making is to reorganise the ECBILT input files to use NetCDF
format files instead of the range of custom binary formats currently
used.  This would make the initialisation files portable between
machines of different endianness.

\paragraph{Model timestepping}

In principle, this also ought to be easy to do -- GENIE uses a single
function call to step that atmosphere along one timestep, and it
should be straightforward to wrap all of the ECBILT timestepping
functionality up into a single routine in that way.  Mismatches
between the GENIE timestep and the ECBILT timestep can be dealt with
by performing multiple ECBILT steps for each GENIE step.  Any
differences between EMBM and ECBILT in terms of the forcings required
from the ocean model can be modified in the wrapper function that
calls the atmosphere model (there shouldn't be any changes here:
surface radiation and moisture fluxes should be all that's needed).

\paragraph{Model coupling}

The ECBILT atmosphere uses a $64 \times 32$ Gaussian grid, so there
may need to be some regridding of fields to and from the regular
GOLDSTEIN grid.  That shouldn't be a problem.

%----------------------------------------------------------------------
\subsection{Model parameterisation issues}

At least one aspect of ECBILT that I've identified may cause trouble
-- it uses an empirical fit to modern atmospheric profiles for its
radiation and atmospheric energy balance (i.e. atmospheric temperature
profile calculation) schemes.  This is true both for the ``original''
ECBILT and the modified version incorporated in LOVECLIM.  At startup
the model reads in atmospheric profiles and associated radiation
coefficients based on (for the LOVECLIM model) the NCEP atmospheric
reanalysis split by region of the globe (including special profiles
for some mountain regions), or (for the ``original'' model)
simulations using a version of the GFDL AGCM.  This dependence on
parameterisations based on modern atmospheric profiles obviously isn't
appropriate for deep paleo simulations where the distribution of land
and ocean is different to modern conditions.

As far as I can tell from a quick search in the literature, the
LOVECLIM EMIC has been used for simulations only as far back as
800\,ka (and that was only one paper -- most applications were more
recent) so this wasn't really a problem before.

Another quick literature survey indicates that there probably aren't
any other EMICs that have been used for paleo simulations that contain
a ``real'' atmospheric GCM: the UVic and McGill models and the CLIMBER
models all use statistical dynamical ``2.5D'' atmosphere models.  The
only ``3D'' GCM that's been coupled into an EMIC seems to be IGCM in
GENIE...  IGCM does have a ``proper'' radiation scheme based on
band absorption coefficients for a selection of absorbing gases,
i.e. it's more or less a ``first principles'' radiation scheme, with
no dependence on parameterisation based on atmospheric profiles.

I don't know whether it would be practical to transplant the IGCM
radiation scheme into ECBILT.  From a survey of the IGCM code, it
certainly looks possible, but I have no idea whether it would work
(IGCM has seven or eight levels in the vertical, ECBILT only three,
which might screw things up), and I have no idea how much it would
slow ECBILT down -- some of the radiation calculations look like they
might be quite onerous.

It's certainly possible that there may be other things like this
hiding within the ECBILT code, but so far the radiation scheme seems
like the biggest impediment.


%======================================================================
\appendix
\section{HadAM3 \texttt{tcszd} comparison results}
\label{sec:hadam3}

\begin{figure}
  \begin{center}
    \includegraphics[width=\textwidth]{../hadam3-comparison/plots/ts-plots}
  \end{center}
  \caption{Seasonal surface temperature for HadAM3 \texttt{tcszd}
    comparison simulation.}
  \label{fig:ts-hadam3}
\end{figure}

\begin{figure}
  \begin{center}
    \includegraphics[width=\textwidth]{../hadam3-comparison/plots/pp-plots}
  \end{center}
  \caption{Seasonal precipitation for HadAM3 \texttt{tcszd} comparison
    simulation.}
  \label{fig:pp-hadam3}
\end{figure}

\begin{figure}
  \begin{center}
    \includegraphics[width=\textwidth]{../hadam3-comparison/plots/evap-plots}
  \end{center}
  \caption{Seasonal evaporation for HadAM3 \texttt{tcszd} comparison
    simulation.}
  \label{fig:evap-hadam3}
\end{figure}

\begin{figure}
  \begin{center}
    \includegraphics[width=\textwidth]{../hadam3-comparison/plots/pmine-plots}
  \end{center}
  \caption{Seasonal precipitation minus evaporation for HadAM3
    \texttt{tcszd} comparison simulation.}
  \label{fig:pmine-hadam3}
\end{figure}

\begin{figure}
  \begin{center}
    \includegraphics[width=\textwidth]{../hadam3-comparison/plots/wind-plots}
  \end{center}
  \caption{Seasonal circulation plots for HadAM3 \texttt{tcszd}
    comparison simulation: arrows show upper level and lower level
    winds, colours show middle atmosphere vertical pressure velocity.}
  \label{fig:wind-hadam3}
\end{figure}

\begin{figure}
  \begin{center}
    \includegraphics[width=\textwidth]{../hadam3-comparison/plots/stress-plots}
  \end{center}
  \caption{Seasonal surface wind stress plots for HadAM3
    \texttt{tcszd} comparison simulation.}
  \label{fig:stress-hadam3}
\end{figure}

\end{document}
